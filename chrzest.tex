\documentclass[11pt, a5paper]{extarticle}
\usepackage{fontspec}
\usepackage[utf8]{luainputenc}
\usepackage[T1]{fontenc}
\usepackage{lmodern}
\usepackage[polish]{babel}
\usepackage{graphicx}
\usepackage[section]{placeins}
\usepackage{float}
\usepackage{array}
\usepackage{tikz}
\usepackage{color}
\usepackage{paracol}
\usepackage{parallel}
\usepackage{pdfcolparallel}
\usepackage{bbding}
\usepackage[labelfont=bf]{caption}
\usepackage[a5paper,top=1.25cm, bottom=1.5cm, left = 1.0cm, right=1.0cm]{geometry}
\usepackage{relsize}

\usepackage{pgfornament}

\newcommand{\textjuni}[1]{{\fontspec{Junicode}#1}}

\newcommand{\versicle}{\textcolor{red}{\textjuni{\char"2123. }}}
\newcommand{\response}{\textcolor{red}{\textjuni{\char"211F. }}}

\newcommand{\SmallCross}{\textcolor{red}{\smaller[1]\CrossMaltese\normalsize}}

\DeclareUnicodeCharacter{01FD}{æ}

\begin{document}

\begin{titlepage}
\begin{center}
\includegraphics{rycina}
\vspace*{0.5cm}
\huge
\textbf{Chrzest \\ Tomasza Michała Kontnego}
\vspace{1.0cm}

	\tikzset{
		pgfornamentstyle/.style={scale=.5}
	}
	\foreach \i in {60} {\expandafter\pgfornament\expandafter{\i}\ }

\vspace{1.5cm}
{\large 21.09.2024}
\end{center}
\end{titlepage}

\begin{center}
\huge

\textbf{Obrzędy chrztu}\\[0.2cm] 
\Large
\textbf{Obrzęd przyjęcia dziecka} \\

\normalsize
\end{center}
\textcolor{red}{Celebrans pyta najpierw rodziców:} \\
\textcolor{red}{K.} \textcolor{black} Drodzy rodzice, jakie imię wybraliście dla swojego dziecka? \\
\textcolor{red}{R.} \textcolor{black} Tomasz Michał.\\
\textcolor{red}{K.} \textcolor{black} O co prosicie Kościół Boży dla Tomasza? \\
\textcolor{red}{R.} \textcolor{black} O wiarę. \\
\indent\textcolor{red} {W powyższym dialogu celebrans może użyć innych słów. Na drugie pytanie rodzice mogą odpowiedzieć innymi słowami, np.: O chrzest, lub: O łaskę Chrystusa, lub: O przyjęcie do Kościoła, lub: O życie wieczne.} \\

\noindent\textcolor{red} {Wtedy celebrans zwraca się do rodziców tymi lub podobnymi słowami:} \\
\textcolor{red}{K.} \textcolor{black} {Drodzy rodzice, prosząc o chrzest dla waszego dziecka, przyjmujecie na siebie obowiązek wychowania go w wierze, aby zachowując Boże przykazania, miłowało Boga i bliźniego, jak nas nauczył Jezus Chrystus. Czy jesteście świadomi tego obowiązku?} \\
\textcolor{red}{R.} \textcolor{black} Jesteśmy tego świadomi.\\
\textcolor{red}{K.} \textcolor{black}{A wy, drodzy chrzestni, czy jesteście gotowi (czy jesteś gotowy, gotowa) pomagać rodzicom tego dziecka w wypełnianiu ich obowiązku?}\\
\textcolor{red}{R.} \textcolor{black}{Jesteśmy gotowi.} \\
\textcolor{red}{K.} \textcolor{black}{Tomaszu Michale wspólnota chrześcijańska przyjmuje cię z wielką radością. W imieniu tej wspólnoty znaczę cię znakiem krzyża. Po mnie wy, rodzice i chrzestni, naznaczcie wasze dziecko znakiem Jezusa Chrystusa, naszego Zbawiciela.}\\
\indent\textcolor{red}{I w milczeniu kreśli znak krzyża na czole dziecka. Potem czynią to rodzice i chrzestni.}\\

\begin{center}
\Large
\textbf{Liturgia Słowa Bożego} \\

\normalsize
\end{center}
\textcolor{red}{Odczytuje się jedno lub dwa czytania biblijne}\\
\begin{center}
\Large
\noindent\textcolor{black}{Modlitwa powszechna}
\end{center}
\textcolor{red}{K.} \textcolor{black}Wezwani przez Pana jako królewskie kapłaństwo, naród święty i lud nabyty przez Niego na własność, prośmy wszechmogącego Boga o miłosierdzie dla tego dziecka, które ma otrzymać łaskę chrztu świętego, dla jego rodziców, chrzestnych i wszystkich ochrzczonych. \\
Wołajmy do Niego: Wysłuchaj nas, Panie. \\

\noindent\versicle Prosimy Cię, abyś przez chrzest święty włączył to dziecko do wspólnoty swojego Kościoła. \\
\response Wysłuchaj nas, Panie. \\
\versicle Prosimy Cię, aby naznaczone krzyżem świętym, przez całe swoje życie odważnie wyznawało Chrystusa, Syna Bożego. \\
\response Wysłuchaj nas, Panie. \\
\versicle Prosimy Cię, aby przez chrzest stało się uczestnikiem Śmierci i Zmartwychwstania Chrystusa. \\
\response Wysłuchaj nas, Panie. \\
\versicle Prosimy Cię, aby to dziecko, pociągnięte słowem i przykładem rodziców i chrzestnych, wzrastało w łasce jako żywy członek Kościoła. \\
\response Wysłuchaj nas, Panie. \\
\versicle Prosimy Cię, abyś we wszystkich tutaj obecnych odnowił łaskę chrztu świętego. \\
\response Wysłuchaj nas, Panie. \\
\versicle Prosimy Cię, aby wszyscy uczniowie Chrystusa, którzy przez chrzest stali się jednym ciałem, zawsze trwali w jednej wierze i miłości. \\
\response Wysłuchaj nas, Panie. \\
\textcolor{red}{Potem celebrans razem z obecnymi wzywa wstawiennictwa Świętych.}
\begin{Parallel}[v]{0.485\textwidth}{0.485\textwidth}
\ParallelLText{
\versicle Sancta Maria , Mater Dei, 
\response ora pro nobis.
\versicle Sánete Ioánnes Baptísta,
\response ora pro nobis.
\versicle Sánete Ioseph,
\response ora pro nobis.
\versicle Sancti Petre et Paule,
\response orate pro nobis.
}
\ParallelRText{
\versicle Święta Maryjo, Matko Boża, \\
\response módl się za nami. \\
\versicle Święty Janie Chrzcicielu, \\
\response módl się za nami. \\
\versicle Święty Józefie, \\
\response  módl się za nami. \\
\versicle Święta Anno \\
\response  módl się za nami. \\
\versicle Święty Tomaszu, \\
\response  módl się za nami. \\
\versicle Święty Michale Archaniele, \\
\response  módl się za nami. \\
\versicle Święci Piotrze i Pawle, \\
\response  módlcie się za nami. \\
\versicle Wszyscy Święci i Święte Boże, \\
\response  módlcie się za nami. \\
}
\end{Parallel}
\end{document}