\documentclass[10pt, a5paper]{extarticle}
\usepackage{fontspec}
\usepackage[utf8]{luainputenc}
\usepackage[T1]{fontenc}
\usepackage{lmodern}
\usepackage[polish]{babel}
\usepackage{graphicx}
\usepackage[section]{placeins}
\usepackage{float}
\usepackage{array}
\usepackage{tikz}
\usepackage{color}
\usepackage{paracol}
\usepackage{parallel}
\usepackage{pdfcolparallel}
\usepackage{bbding}
\usepackage[labelfont=bf]{caption}
\usepackage[a5paper,top=1.25cm, bottom=1.25cm, left = 1.0cm, right=1.0cm]{geometry}
\usepackage{relsize}

\usepackage{pgfornament}

\newcommand{\textjuni}[1]{{\fontspec{Junicode}#1}}

\newcommand{\versicle}{\textcolor{red}{\textjuni{\char"2123. }}}
\newcommand{\response}{\textcolor{red}{\textjuni{\char"211F. }}}

\newcommand{\SmallCross}{\textcolor{red}{\smaller[1]\CrossMaltese\normalsize}}

\DeclareUnicodeCharacter{01FD}{æ}

\begin{document}

\begin{titlepage}
\begin{center}
\includegraphics{rycina}
\vspace*{0.5cm}
\huge
\textbf{Chrzest \\ Tomasza Michała Kontnego}
\vspace{1.0cm}

	\tikzset{
		pgfornamentstyle/.style={scale=.5}
	}
	\foreach \i in {60} {\expandafter\pgfornament\expandafter{\i}\ }

\vspace{1.5cm}
{\large 21.09.2024}
\end{center}
\end{titlepage}

\begin{center}
\huge

\textbf{Obrzędy chrztu}\\[0.2cm] 
\Large
\textbf{Obrzęd przyjęcia dziecka} \\

\normalsize
\end{center}
\textcolor{red}{Celebrans pyta najpierw rodziców:} \\
\textcolor{red}{K.} \textcolor{black} Drodzy rodzice, jakie imię wybraliście dla swojego dziecka? \\
\textcolor{red}{R.} \textcolor{black} Tomasz Michał.\\
\textcolor{red}{K.} \textcolor{black} O co prosicie Kościół Boży dla Tomasza Michała? \\
\textcolor{red}{R.} \textcolor{black} O wiarę. \\
\indent\textcolor{red} {W powyższym dialogu celebrans może użyć innych słów. Na drugie pytanie rodzice mogą odpowiedzieć innymi słowami, np.: O chrzest, lub: O łaskę Chrystusa, lub: O przyjęcie do Kościoła, lub: O życie wieczne.} \\

\noindent\textcolor{red} {Wtedy celebrans zwraca się do rodziców tymi lub podobnymi słowami:} \\
\textcolor{red}{K.} \textcolor{black} {Drodzy rodzice, prosząc o chrzest dla waszego dziecka, przyjmujecie na siebie obowiązek wychowania go w wierze, aby zachowując Boże przykazania, miłowało Boga i bliźniego, jak nas nauczył Jezus Chrystus. Czy jesteście świadomi tego obowiązku?} \\
\textcolor{red}{R.} \textcolor{black} Jesteśmy tego świadomi.\\
\textcolor{red}{K.} \textcolor{black}{A wy, drodzy chrzestni, czy jesteście gotowi (czy jesteś gotowy, gotowa) pomagać rodzicom tego dziecka w wypełnianiu ich obowiązku?}\\
\textcolor{red}{R.} \textcolor{black}{Jesteśmy gotowi.} \\
\textcolor{red}{K.} \textcolor{black}{Tomaszu Michale wspólnota chrześcijańska przyjmuje cię z wielką radością. W imieniu tej wspólnoty znaczę cię znakiem krzyża. Po mnie wy, rodzice i chrzestni, naznaczcie wasze dziecko znakiem Jezusa Chrystusa, naszego Zbawiciela.}\\
\indent\textcolor{red}{I w milczeniu kreśli znak krzyża na czole dziecka. Potem czynią to rodzice i chrzestni.}
\begin{center}
\Large
\textbf{Liturgia Słowa Bożego} \\

\normalsize
\end{center}
\textcolor{red}{Odczytuje się jedno lub dwa czytania biblijne}
\begin{center}
\Large
\noindent\textcolor{black}{Modlitwa powszechna}
\end{center}
\textcolor{red}{K.} \textcolor{black}Wezwani przez Pana jako królewskie kapłaństwo, naród święty i lud nabyty przez Niego na własność, prośmy wszechmogącego Boga o miłosierdzie dla tego dziecka, które ma otrzymać łaskę chrztu świętego, dla jego rodziców, chrzestnych i wszystkich ochrzczonych. \\
Wołajmy do Niego: Wysłuchaj nas, Panie.

\noindent\versicle Prosimy Cię, abyś przez chrzest święty włączył to dziecko do wspólnoty swojego Kościoła. \\
\response Wysłuchaj nas, Panie. \\
\versicle Prosimy Cię, aby naznaczone krzyżem świętym, przez całe swoje życie odważnie wyznawało Chrystusa, Syna Bożego. \\
\response Wysłuchaj nas, Panie. \\
\versicle Prosimy Cię, aby przez chrzest stało się uczestnikiem Śmierci i Zmartwychwstania Chrystusa. \\
\response Wysłuchaj nas, Panie. \\
\versicle Prosimy Cię, aby to dziecko, pociągnięte słowem i przykładem rodziców i chrzestnych, wzrastało w łasce jako żywy członek Kościoła. \\
\response Wysłuchaj nas, Panie. \\
\versicle Prosimy Cię, abyś we wszystkich tutaj obecnych odnowił łaskę chrztu świętego. \\
\response Wysłuchaj nas, Panie. \\
\versicle Prosimy Cię, aby wszyscy uczniowie Chrystusa, którzy przez chrzest stali się jednym ciałem, zawsze trwali w jednej wierze i miłości. \\
\response Wysłuchaj nas, Panie. \\
\textcolor{red}{Potem celebrans razem z obecnymi wzywa wstawiennictwa Świętych.} 
\begin{Parallel}[v]{0.485\textwidth}{0.485\textwidth}
\ParallelLText{
\noindent\versicle Sancta Maria, Mater Dei, \\
\response ora pro nobis.\\
\versicle Sánete Ioánnes Baptísta, \\
\response ora pro nobis. \\
\versicle Sánete Ioseph, \\
\response ora pro nobis. \\
\versicle Sancti Petre et Paule, \\
\response orate pro nobis. \\
\versicle Sancta Anna, \\
\response ora pro nobis. \\
\versicle Sancte Thoma, \\
\response ora pro nobis. \\
\versicle Sancte Michael, \\
\response ora pro nobis. \\
\versicle Omnes Sancti et Sanctæ Dei, \\
\response orate pro nobis.
}
\ParallelRText{
\noindent\versicle Święta Maryjo, Matko Boża, \\
\response módl się za nami. \\
\versicle Święty Janie Chrzcicielu, \\
\response módl się za nami. \\
\versicle Święty Józefie, \\
\response  módl się za nami. \\
\versicle Święci Piotrze i Pawle, \\
\response  módlcie się za nami. \\
\versicle Święta Anno \\
\response  módl się za nami. \\
\versicle Święty Tomaszu, \\
\response  módl się za nami. \\
\versicle Święty Michale, \\
\response  módl się za nami. \\
\versicle Wszyscy Święci i Święte Boże, \\
\response  módlcie się za nami.
}
\end{Parallel}
\begin{center}
\Large
\noindent\textcolor{black}{Modlitwa z egzorcyzmem i włożenie ręki}
\end{center}
\textcolor{red}{Po zakończeniu wezwań do Świętych celebrans mówi:}
\begin{Parallel}[v]{0.485\textwidth}{0.485\textwidth}
\ParallelLText{
Omnipotens sempitèrne Deus,
qui Filium tuum in mundum misisti ,
ut Satanás, spíritus nequítiae, a nobis expélleret potestátem,
et hominem, eréptum e ténebris,
in admirábile lucis tuae regnum transférret:
te supplices exoramus,
ut hunc párvulum,
ab originali labe solutum,
tuae templum perfícias maiestátis,
et Spiritum Sanctum in eo habitare indúlgeas.
Per Christum Dóminum nostrum. \\
\response Amen.
}
\ParallelRText{
Wszechmogący, wieczny Boże, Ty posłałeś na świat swojego Syna, aby oddalił od nas moc szatana, ducha nieprawości, a człowieka wyrwanego z ciemności przeniósł do przedziwnego królestwa Twojej światłości; pokornie Cię błagamy, abyś to dziecko uwolnił od grzechu pierworodnego, uczynił je swoją świątynią i mieszkaniem Ducha Świętego. Przez Chrystusa, Pana naszego. \\
\response Amen.
}
\end{Parallel}
\textcolor{red}{Celebrans mówi:}

\begin{Parallel}[v]{0.485\textwidth}{0.485\textwidth}
\ParallelLText{
Múniat te virtus Christi Salvatóris,
qui vivit et regnai in saécula saeculórum. \\
\response Amen.
}
\ParallelRText{
Niech cię broni moc Chrystusa Zbawiciela,
który żyje i króluje na wieki wieków. \\
\response Amen.
}
\end{Parallel}

\textcolor{red}{Po czym celebrans w milczeniu kładzie rękę na dziecku.}
\textcolor{red}{Następnie wszyscy udają się do chrzcielnicy lub do prezbiterium, jeśli chrztu tam się udziela.}

\begin{center}
\Large
\textbf{Liturgia Sakramentu} \\

\normalsize
\end{center}

\textcolor{red}{Po przyjściu do chrzcielnicy celebrans w krótkich słowach przypomina uczestnikom postanowienie Boga, który chciał duszę i ciało człowieka uświęcić przez wodę. Może to powiedzieć w następujący sposób:} \\
\textcolor{red}{K.} \textcolor{black} Módlmy się, aby Bóg wszechmogący obdarzył to dziecko nowym życiem z wody i z Ducha Świętego. \\

\begin{center}
\Large
\noindent\textcolor{black}{Poświęcenie wody albo modlitwa dziękczynna nad wodą}
\end{center}

\begin{Parallel}[v]{0.485\textwidth}{0.485\textwidth}
\ParallelLText{
Deus, qui invisibili poténtia
per sacramentórum signa miràbilem operàris efféctum,
et creatùram aquae multis modis praeparàsti,
ut Baptismi gràtiam demonstràret;
Deus, cuius Spiritus
super aquas inter ipsa mundi primórdia ferebàtur,
ut i am tunc virtùtem sanctificàndi aquàrum natura conciperet;
Deus, qui regeneratiónis spéciem
in ipsa dilùvii effusióne signàsti,
ut unius eiusdémque eleménti mystério
et finis esset vitiis et origo virtùtum;
Deus, qui Abrahae filios
per mare Rubrum sicco vestigio transire fecisti,
ut plebs, a Pharaónis servitùte liberata,
pópulum baptizatórum praefiguràret;
Deus, cuius Filius , in aqua Iordànis a Ioànne baptizàtus,
Sancto Spiritu est inùnctus,
et, in cruce pendens,
una cum sanguine aquam de latere suo prodùxit,
ac, post resurrectiónem suam, discipulis iussit:
« Ite, docéte omnes gentes,
baptizàntes eos in nòmine Patris et Filii et Spiritus Sancti »:
Réspice in fàciem Ecclésiae tuae,
eique dignàre fontem Baptismatis aperire.
Sumat haec aqua Unigèniti tui gràtiam de Spiritu Sancto,
ut homo, ad imàginem tuam cónditus,
sacraménto Baptismatis
a cunctis squalóribus vetustàtis ablùtus,
in novam infàntiam
ex aqua et Spiritu Sancto resùrgere mereàtur. \\ \\ \\
\textcolor{red}{Celebrans manu dextera tangit aquam et pergit:}\\
Descendat, quaesumus, Domine, in hanc plenitudinem fontis
per Filium tuum virtus Spiritus Sancti,
ut omnes, cum Christo consepulti per Baptismum in mortem,
ad vitam cum ipso resiirgant.
Per Christum Dominum nostrum. \\
\response Amen.
}
\ParallelRText{
Boże, Ty niewidzialną mocą
dokonujesz rzeczy niezwykłych przez sakramentalne znaki.
Ty w ciągu dziejów zbawienia
przygotowałeś wodę przez Ciebie stworzoną,
aby wyrażała łaskę chrztu świętego.
Na początku świata Twój Duch unosił się nad wodami,
aby już wtedy woda nabrała mocy uświęcania.
Boże, Ty nawet w~wodach potopu
dałeś nam obraz odrodzenia,
bo ten sam żywioł
położył kres występkom i dał początek cnotom.
Boże, Ty sprawiłeś, że synowie Abrahama
przeszli po suchym dnie Morza Czerwonego,
aby naród wyzwolony z niewoli faraona
stał się obrazem przyszłej społeczności ochrzczonych.
Boże, Twój Syn, ochrzczony
przez Jana w wodach Jordanu,
został namaszczony Duchem Świętym,
a gdy wisiał na krzyżu, z~Jego boku wypłynęła krew i woda,
po swoim zaś Zmartwychwstaniu nakazał uczniom:
„Idźcie i nauczajcie wszystkie narody
udzielając im chrztu w imię Ojca i Syna, i Ducha Świętego”.
Wejrzyj na swój Kościół
i racz mu otworzyć źródło chrztu świętego.
Niechaj ta woda przez Ducha Świętego otrzyma łaskę
Twojego Jednorodzonego Syna,
aby człowiek stworzony na Twoje podobieństwo
i przez sakrament chrztu obmyty z~wszelkich brudów grzechu,
odrodził się z wody i z Ducha Świętego
do nowego życia dziecka Bożego. \\
\textcolor{red}{Celebrans prawą ręką dotyka wody i mówi:} \\
Prosimy Cię, Panie, niech przez Twojego Syna
zstąpi na tę wodę moc Ducha Świętego,
aby wszyscy, przez chrzest pogrzebani
razem z Chrystusem w śmierci,
z Nim też powstali do nowego życia.
Przez Chrystusa, Pana naszego. \\
\response Amen.
}
\end{Parallel}


\begin{center}
\Large
\noindent\textcolor{black}{Wyrzeczenie się zła i wyznanie wiary}
\end{center}

\textcolor{red}{Celebrans przemawia do rodziców i chrzestnych:}

Drodzy rodzice i chrzestni, przyniesione przez was dziecko przez sakrament chrztu od miłującego Boga otrzyma nowe życie z wody i z Ducha Świętego. Starajcie się wychować je w wierze tak, aby zachować w nim to Boże życie od skażenia grzechem i umożliwić jego ustawiczny rozwój. \\
Jeśli więc, kierując się wiarą, jesteście gotowi podjąć się tego zadania, to wspominając swój własny chrzest, wyrzeknijcie się grzechu i wyznajcie wiarę w Jezusa Chrystusa. Jest to wiara Kościoła, w której wasze dziecko otrzymuje chrzest.\\
\textcolor{red}{K.} Czy wyrzekacie się grzechu, aby żyć w wolności dzieci Bożych?  \\
\textcolor{red}{R.} Wyrzekamy się. \\
\textcolor{red}{K.} Czy wyrzekacie się wszystkiego, co prowadzi do zła, aby was grzech nie opanował?  \\
\textcolor{red}{R.} Wyrzekamy się. \\
\textcolor{red}{K.} Czy wyrzekacie się szatana, który jest głównym sprawcą grzechu?  \\
\textcolor{red}{R.} Wyrzekamy się. \\

\textcolor{red}{Następnie celebrans przyjmuje od rodziców i chrzestnych potrójne wyznanie wiary:} \\
\textcolor{red}{K.} Czy wierzycie w Boga, Ojca wszechmogącego, Stworzyciela nieba i ziemi?  \\
\textcolor{red}{R.} Wierzymy. \\
\textcolor{red}{K.} Czy wierzycie w Jezusa Chrystusa, Jego Syna jedynego, a naszego Pana, narodzonego z Maryi Dziewicy, umęczonego i pogrzebanego, który powstał z martwych i zasiada po prawicy Ojca?  \\
\textcolor{red}{R.} Wierzymy. \\
\textcolor{red}{K.} Czy wierzycie w Ducha Świętego, święty Kościół powszechny, obcowanie Świętych, odpuszczenie grzechów, zmartwychwstanie ciała i życie wieczne?  \\
\textcolor{red}{R.} Wierzymy.

\textcolor{red}{Do tego wyznania dołącza się celebrans i wszyscy zgromadzenia:} \\
\textcolor{red}{K.} Taka jest nasza wiara. Taka jest wiara Kościoła, której wyznawanie jest naszą chlubą, w Chrystusie Jezusie, Panu naszym. \\
\response Amen.

\begin{center}
\Large
\noindent\textcolor{black}{Chrzest}
\end{center}

\textcolor{red}{Celebrans zaprasza rodzinę, aby podeszła do chrzcielnicy, i pyta rodziców i chrzestnych:} \\
\textcolor{red}{K.} Czy chcecie, aby Tomasz Michał otrzymał chrzest w wierze Kościoła, którą przed chwilą wspólnie wyznaliśmy? \\
\textcolor{red}{R.} Chcemy.

\begin{center}
\textcolor{red}{Teraz celebrans chrzci dziecko, mówiąc:}
\end{center}

\begin{Parallel}[v]{0.485\textwidth}{0.485\textwidth}
\ParallelLText{
Thoma Michael, ego te baptizo in nòmine Patris,
}
\ParallelRText{
Tomaszu Michale, ja ciebie chrzczę w imię Ojca
}
\end{Parallel}

\begin{center}
\textcolor{red}{polewa je wodą po raz pierwszy.}
\end{center}

\begin{Parallel}[v]{0.485\textwidth}{0.485\textwidth}
\ParallelLText{
et Filii,
}
\ParallelRText{
i Syna
}
\end{Parallel}

\begin{center}
\textcolor{red}{polewa je wodą po raz drugi.}
\end{center}

\begin{Parallel}[v]{0.485\textwidth}{0.485\textwidth}
\ParallelLText{
et Spiritus Sancti.
}
\ParallelRText{
i Ducha Świętego
}
\end{Parallel}

\begin{center}
\textcolor{red}{polewa je wodą po raz trzeci.}
\end{center}


\textcolor{red}{Wypada, aby po chrzcie dziecka uczestnicy wypowiadali lub śpiewali krótką aklamację:}\\
Chwała Ojcu i Synowi, i Duchowi Świętemu. Jak była na początku, teraz i zawsze, i na wieki wieków. Amen.

\begin{center}
\Large
\noindent\textcolor{black}{Obrzędy wyjaśniające}
\end{center}

\begin{center}
\textcolor{red}{Namaszczenie krzyżmem świętym.}
\end{center}

\begin{Parallel}[v]{0.485\textwidth}{0.485\textwidth}
\ParallelLText{
Deus omnipotens, Pater Dòmini nostri Iesu Christi, qui
te a peccato liberâvit et regenerâvit ex aqua et Spiritu Sancto,
ipse te linit chrismate salutis, ut, eius aggregâtus pópulo,
Christi sacerdótis, prophétae et regis membrum permâneas in
vitam aetérnam. \\
\response Amen.
}
\ParallelRText{
Bóg wszechmogący, Ojciec naszego Pana, Jezusa Chrystusa, który cię uwolnił od grzechu i odrodził z wody i z Ducha Świętego, On sam namaszcza ciebie krzyżmem zbawienia, abyś włączony(a) do ludu Bożego, wytrwał(a) w jedności z Chrystusem Kapłanem, Prorokiem i Królem na życie wieczne.\\
\response Amen.
}
\end{Parallel}

\textcolor{red}{Potem celebrans w milczeniu namaszcza dziecko krzyżmem św. na szczycie głowy.}

\begin{center}
\textcolor{red}{Włożenie białej szaty.}
\end{center}

\begin{Parallel}[v]{0.485\textwidth}{0.485\textwidth}
\ParallelLText{
Thoma Michael nova creatura factus es et Christum induisti. Vestís
haec càndida sit tibi signum dignitatis, quam, tuórum verbo
et exémplo propinquórum adiútus, immaculátam pérferas in
vitam aetérnam. \\
\response Amen.
}
\ParallelRText{
Tomaszu Michale, stałeś się nowym stworzeniem i przyoblokłeś się w Chrystusa, dlatego otrzymujesz białą szatę. Niech twoi bliscy słowem i przykładem pomagają ci zachować godność dziecka Bożego, nieskalaną aż po życie wieczne.\\
\response Amen.
}
\end{Parallel}

\textcolor{red}{Nakłada się dziecku białą szatę.}

\begin{center}
\textcolor{red}{Wręczenie zapalonej świecy}
\end{center}
\textcolor{red}{Celebrans bierze świecę paschalną i mówi:}

\begin{Parallel}[v]{0.485\textwidth}{0.485\textwidth}
\ParallelLText{
Lumen Christi accipite.
}
\ParallelRText{
Celebrans bierze świecę paschalną i mówi:
}
\end{Parallel}

\textcolor{red}{Przedstawiciel rodziny zapala świecę dziecka od świecy paschalnej.}

\begin{Parallel}[v]{0.485\textwidth}{0.485\textwidth}
\ParallelLText{
Vobis, paréntibus et patrino (vel patrinis), lumen hoc
concréditur fovéndum, ut párvulus iste, a Christo illuminátus,
tamquam fílius lucis indesinénter ámbulet et, in fide
persevérans, ad veniènti Dòmino occúrrere váleat cum òmnibus
Sanctis in aula caelésti.
}
\ParallelRText{
Podtrzymywanie tego światła powierza się wam, rodzice i chrzestni, aby wasze dziecko, oświecone przez Chrystusa, postępowało zawsze jak dziecko światłości, a trwając w wierze, mogło wyjść na spotkanie przychodzącego Pana razem z wszystkimi Świętymi w niebie.
}
\end{Parallel}

\begin{center}
\Large
\noindent\textcolor{black}{Zakończenie obrzędu}
\end{center}

\textcolor{red}{Celebrans stojąc przed ołtarzem, przemawia do rodziców i chrzestnych oraz do wszystkich obecnych w ten lub podobny sposób:}

\begin{Parallel}[v]{0.485\textwidth}{0.485\textwidth}
\ParallelLText{
Fratres dilectissimi: pàrvulus iste, qui, per Baptismum regeneràtus,
filius Dei nominàtur et est, plenitùdinem Spiritus
Sancti per Confirmatiónem recipiet et, ad altare Dòmini accédens,
pàrticeps fiet mensae sacrificii eius ac Deum in mèdio
Ecclèsia? Patrem vocàbit. Nunc nòmine eius, in spiritu adoptiónis
filiórum, quem omnes accépimus, simul orèmus uti Dóminus nos dócuit orare.
}
\ParallelRText{
Najmilsi, to dziecko, odrodzone przez chrzest święty, nazywa się dzieckiem Bożym i jest nim rzeczywiście. W sakramencie bierzmowania otrzyma pełnię Ducha Świętego. Przystępując zaś do ołtarza Pańskiego, stanie się uczestnikiem Jego Uczty Ofiarnej i razem ze zgromadzonym Kościołem będzie Boga nazywać Ojcem. My wszyscy także jesteśmy dziećmi Bożymi, dlatego w imieniu tego dziecka módlmy się wspólnie, tak jak nas nauczył Pan Jezus.
}
\end{Parallel}

\textcolor{red}{Wszyscy razem z celebransem mówią:}

\begin{Parallel}[v]{0.485\textwidth}{0.485\textwidth}
\ParallelLText{
Pater noster, qui es in caelis:
sanctificétur nomen tuum;
advéniat regnum tuum;
fiat voluntas tua, sicut in caelo, et in terra.
Panem nostrum cotidiànum da nobis hódie;
et dimitte nobis débita nostra,
sicut et nos dimittimus debitóribus nostris;
et ne nos indùcas in tentatiónem;
sed libera nos a malo.
}
\ParallelRText{
Ojcze nasz, któryś jest w niebie,
święć się imię Twoje,
przyjdź królestwo Twoje,
bądź wola Twoja
jako w niebie, tak i na ziemi.
Chleba naszego powszedniego daj nam dzisiaj
i odpuść nam nasze winy,
jako i my odpuszczamy naszym winowajcom.
I nie wódź nas na pokuszenie,
ale nas zbaw ode złego. Amen.
}
\end{Parallel}

\begin{center}
\Large
\noindent\textcolor{black}{Błogosławieństwo}
\end{center}

\textcolor{red}{Następnie celebrans błogosławi matkę trzymającą na rękach swoje dziecko, a także ojca i wszystkich obecnych.}

\begin{Parallel}[v]{0.485\textwidth}{0.485\textwidth}
\ParallelLText{
Dóminus Deus omnipotens, qui per Filium suum natum ex
María Vírgine Christianas laetíficat matres aetérnae spe
vitas, quae suis affúlget infántibus, dignétur matrem huius
benedícere infántis, ut, quae de sobóle grátias nunc agit
accépta, perpètuo cum ipsa in gratiárum máneat actióne,
in Christo Iesu Dòmino nostro.\\
\response Amen.
}
\ParallelRText{
Pan Bóg wszechmogący, który przez swojego Syna, narodzonego z Maryi Dziewicy, niesie chrześcijańskim matkom radość z tego, że ich dzieciom zajaśniała nadzieja życia wiecznego, niechaj błogosławi matce tego dziecka i tak, jak teraz dziękuje ona Bogu za otrzymane potomstwo, niech zawsze trwa w dziękczynieniu razem ze swoim dzieckiem, w Chrystusie Jezusie, Panu naszym. \\
\response Amen.
}
\end{Parallel}

\begin{Parallel}[v]{0.485\textwidth}{0.485\textwidth}
\ParallelLText{
Dóminus Deus omnipotens, qui vit am terrénam
largítur et caeléstem, patrem huius infántis benedicat,
ut, una cum coniuge sua, verbo et exémplo proli priórem
se fídei testem exhibeat, in Christo Iesu Dòmino nostro.\\
\response Amen.
}
\ParallelRText{
Pan Bóg wszechmogący, dawca życia doczesnego i wiecznego, niechaj błogosławi ojcu tego dziecka, aby słowem i przykładem dawał swojemu dziecku pierwsze świadectwo wiary w Chrystusie Jezusie, Panu naszym. \\
\response Amen.
}
\end{Parallel}

\begin{Parallel}[v]{0.485\textwidth}{0.485\textwidth}
\ParallelLText{
Dóminus Deus omnipotens, qui nos ex aqua et
Spiritu Sancto in vitam regenerávit aetérnam, hos fidéles
suos munificus benedicat, ut semper et ubique vivida
sint membra pópuli sui, et pacem suam òmnibus hic
praeséntibus largiátur, in Christo Iesu Dòmino nostro.\\
\response Amen.
}
\ParallelRText{
Pan Bóg wszechmogący, który nas odrodził z wody i z Ducha Świętego na życie wieczne, niechaj swoim wiernym, tu zebranym, udzieli obfitego błogosławieństwa, aby zawsze i wszędzie trwali we wspólnocie ludu Bożego. Niechaj też wszystkich tu obecnych obdarzy swoim pokojem, w Chrystusie Jezusie, Panu naszym. \\
\response Amen.
}
\end{Parallel}

\begin{Parallel}[v]{0.485\textwidth}{0.485\textwidth}
\ParallelLText{
Benedicat vos omnipotens Deus,
Pater, et Filius , \SmallCross  et Spiritus Sanctus..\\
\response Amen. \\
\versicle Ite in pace.\\
\response Deo grátias.
}
\ParallelRText{
Niech was błogosławi Bóg wszechmogący, Ojciec i Syn, \SmallCross i Duch Święty. \\
\response Amen.\\
\versicle Idźcie w pokoju.\\
\response Bogu dzięki.
}
\end{Parallel}

\end{document}